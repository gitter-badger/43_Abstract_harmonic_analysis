\documentclass[12pt]{article}
\usepackage{pmmeta}
\pmcanonicalname{PontryaginDuality}
\pmcreated{2013-03-22 17:42:42}
\pmmodified{2013-03-22 17:42:42}
\pmowner{asteroid}{17536}
\pmmodifier{asteroid}{17536}
\pmtitle{Pontryagin duality}
\pmrecord{7}{40156}
\pmprivacy{1}
\pmauthor{asteroid}{17536}
\pmtype{Theorem}
\pmcomment{trigger rebuild}
\pmclassification{msc}{43A40}
\pmclassification{msc}{22B05}
\pmclassification{msc}{22D35}
\pmsynonym{Pontrjagin duality}{PontryaginDuality}
\pmsynonym{Pontriagin duality}{PontryaginDuality}
\pmrelated{DualityInMathematics}
\pmdefines{Pontryagin dual}
\pmdefines{Pontrjagin dual}
\pmdefines{Pontriagin dual}
\pmdefines{dual of an abelian group}
\pmdefines{character}

% this is the default PlanetMath preamble.  as your knowledge
% of TeX increases, you will probably want to edit this, but
% it should be fine as is for beginners.

% almost certainly you want these
\usepackage{amssymb}
\usepackage{amsmath}
\usepackage{amsfonts}

% used for TeXing text within eps files
%\usepackage{psfrag}
% need this for including graphics (\includegraphics)
%\usepackage{graphicx}
% for neatly defining theorems and propositions
%\usepackage{amsthm}
% making logically defined graphics
%%%\usepackage{xypic}

% there are many more packages, add them here as you need them
\usepackage{amsthm}


% define commands here

\begin{document}
\section{Pontryagin dual}

Let $G$ be a locally compact abelian \PMlinkname{group}{TopologicalGroup} and $\mathbb{T}$ the \PMlinkname{1-torus}{NTorus}, i.e. the unit circle in $\mathbb{C}$.

{\bf Definition -} A continuous homomorphism $G \longrightarrow \mathbb{T}$ is called a {\bf character} of $G$. The set of all characters is called the {\bf Pontryagin dual} of $G$ and is denoted by $\hat{G}$.

Under pointwise multiplication $\hat{G}$ is also an abelian group. Since $\hat{G}$ is a group of functions we can make it a topological group under the compact-open topology (topology of convergence on compact sets).

\section{Examples}
\begin{itemize}
\item $\hat{\mathbb{Z}} \cong \mathbb{T}$, via $n \mapsto z^n$ with $z \in \mathbb{T}$.
\item $\hat{\mathbb{T}} \cong \mathbb{Z}$, via $z \mapsto z^n$ with $n \in \mathbb{Z}$.
\item $\hat{\mathbb{R}} \cong \mathbb{R}$, via $t \mapsto e^{ist}$ with $s \in \mathbb{R}$.
\end{itemize}

\section{Properties}

The following are some important \PMlinkescapetext{properties} of the dual group:

{\bf Theorem -} Let $G$ be a locally compact abelian group. We have that
\begin{itemize}
\item $\hat{G}$ is also locally compact.
\item $\hat{G}$ is second countable if and only if $G$ is second countable.
\item $\hat{G}$ is compact if and only if $G$ is discrete.
\item $\hat{G}$ is discrete if and only if $G$ is compact.
\item $\widehat{(\oplus_{i \in J} G_i)} \cong \oplus_{i \in J} \hat{G_i}$ for any finite set $J$. This isomorphism is natural.
\end{itemize}

\section{Pontryagin duality}

Let $f:G \longrightarrow H$ be a continuous homomorphism of locally compact abelian groups. We can associate to it a canonical map $\hat{f}:\hat{H} \longrightarrow \hat{G}$ defined by
\begin{align*}
\hat{f}(\phi)\,(s):= \phi(f(s))\;, \qquad\qquad \phi \in \hat{H},\; s \in G
\end{align*}
This canonical construction preserves identity mappings and compositions, i.e. the dualization process $\hat{\;}$ is a functor:

{\bf Theorem -} The dualization $\hat{\;}:\mathord{\mathbf{LcA}} \longrightarrow \mathord{\mathbf{LcA}} $ is a contravariant functor from the category of locally compact abelian groups to itself.

\section{Isomorphism with the second dual}

Although in general there is not a canonical identification of $G$ with its dual $\hat{G}$, there is a natural isomorphism between $G$ and its dual's dual $\hat{\hat{G}}$:

{\bf Theorem -} The map $G \longrightarrow \hat{\hat{G}}$ defined by $s \mapsto \hat{\hat{s}}$, where $\hat{\hat{s}}(\phi) := \phi(s)$, is a natural isomorphism between $G$ and $\hat{\hat{G}}$.

\section{Applications}

The study of dual groups allows one to visualize Fourier series, Fourier transforms and discrete Fourier transforms from a more abstract and unified view-point, providing the \PMlinkescapetext{basis} for a general definition of Fourier transform. Thus, dual groups and Pontryagin duality are the \PMlinkescapetext{foundations} of the \PMlinkescapetext{theory} of abstract abelian harmonic analysis.
%%%%%
%%%%%
\end{document}
