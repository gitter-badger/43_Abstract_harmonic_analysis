\documentclass[12pt]{article}
\usepackage{pmmeta}
\pmcanonicalname{ProofOfAngleSumIdentities}
\pmcreated{2013-05-28 14:35:29}
\pmmodified{2013-05-28 14:35:29}
\pmowner{rspuzio}{6075}
\pmmodifier{rspuzio}{6075}
\pmtitle{proof of angle sum identities}
\pmrecord{15}{39798}
\pmprivacy{1}
\pmauthor{rspuzio}{6075}
\pmtype{Proof}
\pmcomment{trigger rebuild}
\pmclassification{msc}{43-00}
\pmclassification{msc}{51-00}
\pmclassification{msc}{42-00}
\pmclassification{msc}{33B10}

% this is the default PlanetMath preamble.  as your knowledge
% of TeX increases, you will probably want to edit this, but
% it should be fine as is for beginners.

% almost certainly you want these
\usepackage{amssymb}
\usepackage{amsmath}
\usepackage{amsfonts}

% used for TeXing text within eps files
%\usepackage{psfrag}
% need this for including graphics (\includegraphics)
%\usepackage{graphicx}
% for neatly defining theorems and propositions
\usepackage{amsthm}
% making logically defined graphics
\usepackage{xypic}

% there are many more packages, add them here as you need them

% define commands here
\newtheorem{thm}{Theorem}
\begin{document}
We will derive the angle sum identities for the various trigonometric
functions here.  We begin by deriving the identity for the sine by means 
of a geometric argument and then obtain the remaining identities by
algebraic manipulation.

\begin{thm}
\[
\sin (x + y) = \sin (x) \cos (y) + \cos (x) \sin (y)
\]
\end{thm}

\begin{proof}
Let us make the restrictions $0^\circ < x < 90^\circ$ and 
$0^\circ < y < 90^\circ$ for the time being.  Then we may draw a
triangle $ABC$ such that $\angle CAB = x$ and $\angle ABF = y$:
\[
\begin{xy}
,(0,0)
;(70,0)**@{-}
;(40,30)**@{-}
;(0,0)**@{-}
,(-2,-2)*{A}
,(72,-2)*{B}
,(40,32)*{C}
\end{xy}
\]
Since the angles of a triangle add up to $180^\circ$, we must have
$\angle BCA = 180^\circ - x - y$, so we have $\sin (\angle BCA) =
\sin (180^\circ - x - y) = \sin (x + y)$.

We now draw perpendiculars two different ways in order to derive ratios.
First, we drop a perpendicular $AD$ from $C$ to $AB$:
\[
\begin{xy}
,(0,0)
;(70,0)**@{-}
;(40,30)**@{-}
;(0,0)**@{-}
,(40,30)
;(40,0)**@{-}
,(-2,-2)*{A}
,(72,-2)*{B}
,(40,32)*{C}
,(40,-2)*{D}
\end{xy}
\]
Since $ACD$ and $BCD$ are right triangles we have, by definition,
\[
\cot (\angle CAB) = \overline{AD} / \overline{CD} \qquad
\cot (\angle ABC) = \overline{BD} / \overline{CD} \qquad
\sin (\angle CAB) = \overline{CD} / \overline{AC} .
\]

Second, we draw a perpendicular $AE$ form $A$ to $BC$.  Depending on whether
$x+y < 90^\circ$ or $x+y < 90^\circ$ the point $E$ will or will not lie between
$B$ and $C$, as illustrated below.  (There is also the case $x+y = 90^\circ$, 
but it is trivial.)
\[
\begin{xy}
,(0,0)
;(70,0)**@{-}
;(30,40)**@{-}
;(0,0)**@{-}
,(0,0)
;(35,35)**@{-}
,(-2,-2)*{A}
,(72,-2)*{B}
,(30,42)*{C}
,(36,36)*{E}
\end{xy}
\]
\[
\begin{xy}
,(0,0)
;(70,0)**@{-}
;(35,35)**@{-}
;(0,0)**@{-}
,(0,0)
;(40,30)**@{-}
,(-2,-2)*{A}
,(72,-2)*{B}
,(41,31)*{C}
,(35,37)*{E}
\end{xy}
\]
Either way, $ABE$ and $ACE$ are right triangles, and we have, by definition,
\[
\sin (\angle BCA) = \overline{AE} / \overline{AC} \qquad
\sin (\angle ABC) = \overline{AE} / \overline{AB} .
\]
Combining these ratios, we find that
\[
\sin (\angle BCA) / \sin (\angle ABC) = \overline{AB} / \overline{AC} .
\]

To finish deriving the sum identity, we manipulate the ratios derived above
algebraically and use the fact that $\overline{AD} + \overline{BD} = 
\overline{AB}$:
\begin{align*}
\sin (x + y) = \sin (\angle BCA) &=
\overline{AB} \, \sin (\angle ABC) / \overline{AC} \\ &=
(\overline{AD} + \overline{BD}) \sin (\angle ABC) / \overline{AC} \\ &=
\overline{CD} \left( \cot (\angle CAB) + \cot (\angle ABC) \right) / 
\sin (\angle ABC) \overline{AC}  \\ &=
\sin (\angle CAB) \sin (\angle ABC)
\left( {\cos (\angle CAB) \over \sin (\angle CAB)} +
{\cos (\angle ABC) \over \sin (\angle ABC)} \right) \\ &=
\sin (\angle CAB) \cos (\angle ABC) + 
\cos (\angle CAB) \sin (\angle ABC) \\ &=
\sin (x) \cos (y) + \cos (x) \sin (y)
\end{align*}

To lift the restriction on the range of $x$ and $y$, we use the identities
for complements and negatives of angles.  

\end{proof}

\textbf{Entry under construction}
%%%%%
%%%%%
\end{document}
