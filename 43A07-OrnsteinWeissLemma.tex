\documentclass[12pt]{article}
\usepackage{pmmeta}
\pmcanonicalname{OrnsteinWeissLemma}
\pmcreated{2013-03-22 19:20:24}
\pmmodified{2013-03-22 19:20:24}
\pmowner{Ziosilvio}{18733}
\pmmodifier{Ziosilvio}{18733}
\pmtitle{Ornstein-Weiss lemma}
\pmrecord{5}{42288}
\pmprivacy{1}
\pmauthor{Ziosilvio}{18733}
\pmtype{Theorem}
\pmcomment{trigger rebuild}
\pmclassification{msc}{43A07}

\endmetadata

% this is the default PlanetMath preamble.  as your knowledge
% of TeX increases, you will probably want to edit this, but
% it should be fine as is for beginners.

% almost certainly you want these
\usepackage{amssymb}
\usepackage{amsmath}
\usepackage{amsfonts}

% used for TeXing text within eps files
%\usepackage{psfrag}
% need this for including graphics (\includegraphics)
%\usepackage{graphicx}
% for neatly defining theorems and propositions
%\usepackage{amsthm}
% making logically defined graphics
%%%\usepackage{xypic}

% there are many more packages, add them here as you need them

% define commands here

\begin{document}
\newcommand{\Nset}{\ensuremath{\mathbb{N}}}
\newcommand{\Rset}{\ensuremath{\mathbb{R}}}
\newcommand{\Zset}{\ensuremath{\mathbb{Z}}}

\newtheorem{theorem*}{Theorem}

Let $G$ be a group.
For a fixed $K \subseteq G$,
define the \emph{$K$-boundary} of $U \subseteq G$ as
\begin{equation} \label{eq:bd}
\partial_K U = \left\{
g \in G \mid Kg \cap U, Kg \cap (G \setminus U) \neq \emptyset
\right\} \;.
\end{equation}
Let $\mathcal{PF}(G)$ be the set of finite subsets of $G$.
Call a \emph{F{\o}lner net} for $G$ a net
\begin{math}
\mathcal{X} = \{X_i\}_{i \in \mathcal{I}} \subseteq \mathcal{PF}(G),
\end{math}
$\mathcal{I}$ being a directed set,
such that for every \emph{finite} $K \subseteq G$,
\begin{equation} \label{eq:fo}
\lim_{i \in \mathcal{I}} \frac{|\partial_{K} X_i|}{|X_i|} = 0 \;,
\end{equation}
where the limit is taken in the sense of directed sets.
Recall that $G$ has a F{\o}lner net if and only if $G$ is amenable.

\begin{theorem*}[Ornstein-Weiss lemma]
Let $G$ be an amenable group and
\begin{math}
F : \mathcal{PF}(G) \to \Rset
\end{math}
a subadditive, right-invariant function, that is:
\begin{enumerate}
\item \label{it:sa}
For any two finite subsets $U,V$ of $G$,
\begin{equation} \label{eq:sa}
F(U \cup V) \leq F(U) + F(V) \;.
\end{equation}
\item \label{it:ri}
For any $g \in G$ and finite $U \subseteq G$,
\begin{equation} \label{eq:ri}
F(Ug) = F(U) \;.
\end{equation}
\end{enumerate}
Then for any F{\o}lner net
\begin{math}
\mathcal{X} = \left\{ X_i \right\}_{i \in \mathcal{I}}
\end{math}
on $G$, the limit
\begin{equation} \label{eq:ow}
L = \lim_{i \in \mathcal{I}} \frac{F(X_i)}{|X_i|}
\end{equation}
exists, and does not depend on the choice of $\mathcal{X}$.
\end{theorem*}

The Ornstein-Weiss lemma allows to prove variants of Birkhoff's ergodic theorem
for actions of amenable groups,
rather than only those generated by an invertible, measure invariant map.
Moreover, it shares several similarities
with Fekete's lemma on subadditive functions over the positive integers,
although it is not a complete counterpart.
In fact, putting
\begin{math}
X_n = \{1,\ldots,n\}
\end{math}
determines a F{\o}lner sequence on $\Zset$;
however, if $f : \Nset \to [0,\infty)$ is subadditive,
then $F(U) = f(|U|)$ is right-invariant, but not necessarily subadditive.
(Counterexample: $f(n) = n \,\mathrm{mod}\, 2$, $U = \{1,2\}$, $V = \{2,3\}$.)

\begin{thebibliography}{99}

\bibitem{g}
Gromov, M. (1999)
Topological invariants of dynamical systems and spaces of holomorphic maps, I.
\textit{Math. Phys. Anal. Geom.} \textbf{2}, 323--415.

\bibitem{kf}
Krieger, F. (2007)
Le lemme d'Ornstein-Weiss d'apr\`es Gromov.
In B. Hasselblatt (ed.), \textit{Dynamics, Ergodic Theory, and Geometry}.
Cambridge University Press.

\bibitem{k}
Krieger, F.
The Ornstein-Weiss lemma for discrete amenable groups.
Preprint.


\bibitem{ow}
Ornstein, D.S. and Weiss, B. (1987)
Entropy and isomorphism theorems for actions of amenable groups.
\textit{J. Anal. Math.} \textbf{48}, 1--141.

\end{thebibliography}

%%%%%
%%%%%
\end{document}
